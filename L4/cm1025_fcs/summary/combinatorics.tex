\section{Combinatorics}
\paragraph{Product Rule} Suppose that a procedure can be broken down into a sequence of two tasks. If there are \( n_1 \)  ways to do the first task and for each of these ways of doing the first task, there are \( n_2 \) ways to do the second task; then there are \( n_1 n_2 \)  ways to do the procedure.
\paragraph{Sum Rule} 
If a task can be done either in one of \( n_1 \)  ways or in one of \( n_2 \) ways, where none of the set of \( n_1 \) ways is the same as any of the set of \( n_2 \) ways, then there are \( n_1 + n_2 \) ways to do the task.
\paragraph{Subtraction Rule} 
If a task can be done in either \( n_1 \) ways or \( n_2 \) ways, then the number of ways to do the task is \( n_1 + n_2 \) minus the number of ways to do the task that are common to the two different ways. This is the same as the \textbf{principle of inclusion-exclusion} (see Discrete Mathematics --- Set Theory).
\subsection{Pigeonhole Principle}
If \( N \) objects are placed into \( k \) boxes, then there is \textbf{at least} one box with \textbf{at least} \(\lceil N/k \rceil \) objects.

\textbf{Important:} When calculating the number of objects \( N \)  required to satisfy a specific \(\lceil N/k \rceil \) outcome, keep in mind that you are rounding up \( N/k \). For example, if \( \lceil N/k \rceil = 6 \) and \( k = 4 \), then 
\begin{align*}
	\lceil N/4 \rceil = 6 \Leftrightarrow N/4 &> 5\\
	N &> 20 \\
	N &= 21
\end{align*}
Therefore, when solving the inequality, you can add one to the result for N to arrive at the smallest number that will satisfy the desired \( N/k \) (if that is what is asked). 

\subsection{Permutations}
A permutation of \(n\) different elements is an ordering of the elements such that one element is first, one is second, one is third, and so on.

The number of permutations of \(n\) elements is
\[
	\begin{array}{l}{n \cdot(n-1) \cdot \cdot 4 \cdot 3 \cdot 2 \cdot 1=n !} \\ \end{array}
\]
In other words, there are \( n! \) different ways of ordering \(n\) elements.

The number of permutations of \(n\) elements taken \(r\) at a time \textbf{without repetition}  is

\[
	\nPr{n}{r}=\frac{n !}{(n-r) !}=n(n-1)(n-2) \cdots(n-r+1)
\]

The number or permutations of \( n \) elements taken \( r \) at a time \textbf{with repetition allowed}  inclusion
\[
n^r	
\]

Consider a set of \(n\) objects that has \(n_1\) of one kind of object, \(n_2\) of a second kind, and so on. The number of \textbf{distinguishable permutations} of the \(n\) objects is

\begin{equation}
	\frac{n !}{n_{1} ! \cdot n_{2} ! \cdot n_{3} ! \cdot \cdot \cdot \cdot \cdot n_{k} !}
\end{equation}

This is equivalent to the number of ways \( n \) distinguishable objects can be placed into \( k \) boxes.

\subsection{Combinations}

Combinations consider only the possible sets of objects \emph{regardless} of the order in which the members of the set are arranged.

The number of possible combinations of \(n\) elements taken \(r\) at a time \textbf{without repetition}  is

\begin{equation}
	\nCr{n}{r}=\frac{n !}{(n-r) ! r !} = \frac{\nPr{n}{r}}{r!}
\end{equation}


The number of possible combinations of \(n\) elements taken \(r\) at a time \textbf{with repetition}  is
\[
	\nCr{n}{r}=\frac{(n+r-1)!}{(n-1)!r!}	
\]

\subsection{Binomial Theorem}
The number of \( r \)-combinations from a set with \( n \) elements can be denoted as
\( (\begin{smallmatrix}
	n \\
	r
\end{smallmatrix}) \).
A \textbf{binomial} expression is the sum of two terms. The \textbf{binomial theorem} states that given two variables \( x \) and \( y \), and a nonnegative integer \( n \),
\[
(x + y)^n = \sum_{j=0}^n \begin{pmatrix} n\\j \end{pmatrix} x^{n-j}y^j = 	\begin{pmatrix} n\\0 \end{pmatrix}x^n + \begin{pmatrix} n\\1 \end{pmatrix} x^{n-1}y+ \cdots + \begin{pmatrix} n\\n-1 \end{pmatrix}xy^{n-1}+ \begin{pmatrix} n\\n \end{pmatrix}y^n
\] 

Note that the exponent of \( n \) decreases with each term as the exponent of \( y \) increases with each term. Also note that if \( x \) or \( y \) is negative, then each alternating term will be negative by moving the negative sign from the negative term to the front (e.g. \( x(-y)^3 =  - xy^3  \) ).